% !TEX root = Projektstudie.tex
% Fazit

\section{Fazit}
\label{sec:Fazit}

Während zahlreicher Versuchsfahrten stellt sich heraus, dass eine  genaue Positionierung nur mit Kenntnis der Radgeschwindigkeiten keine möglich ist. Kleinste Unebenheiten oder Verschmutzungen resultieren in durchrutschenden oder abhebenden Rädern und somit in starken Fahrbahnabweichungen. Für den produktiven Betrieb sind daher externe Positionsreferenzen zwingend notwendig. Weitere Ausführungen dazu folgen im Ausblick Kapitel~\ref{sec:Ausblick}.

Neben dem Fahrbahnuntergrund erschweren mechanische Probleme ein reibungsloses Fahren des Mecanum-Roboters: 

\begin{itemize}
    \item{Die Motoren sind zu schwach dimensioniert und verlieren bei abrupten Manövern Schritte.}
    \item{Das treibende Rad des Riementriebs löst sich nach längerem Gebrauch von der Welle und muss mit gezielten Hammerschlägen wieder an seine Position gebracht werden. Dies hat zur Folge, dass die Mecanum-Räder zu nah am Riementrieb liegen und sich regelmäßig mit den überstehenden Schrauben im Riemen verhaken.}
\end{itemize}

Ein Beheben dieser Probleme setzt eine Neukonstruktion der betrachteten Komponenten voraus. Weitere notwendige Änderungen werden neben Optimierungsansätzen und möglichen folgenden Projekten im Ausblick Kapitel~\ref{sec:Ausblick} beschrieben.
