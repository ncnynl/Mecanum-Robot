% !TEX root = Projektstudie.tex
% Hardware

\section{Hardwaremodifikation}
\label{sec:Hardwaremodifikation}
Um die Sicherheit der Benutzer zu gewährleisten, müssen einige Modifikationen an der Verkabelung und am Zusammenbau vorgenommen werden.

\begin{itemize}
    \item{Die Netzanschlussleitung der vier Netzteile wird komplett erneuert. Der Schutzleiter wird angeschlossen. Dabei wird auf eine korrekte Anwendung der Zugentlastungen geachtet.}
    \item{Die Erdung des Schutzleiters wird über Erdungsklemmen auf das Fahrzeuggehäuse gelegt.}
    \item{Die verwendete Schraubenlänge zur Befestigung der Mecanum-Räder war ungenügend. Alle vier Mecanum-Räder sind nun mit längeren M5 Schrauben befestigt.}
\end{itemize}

Die Änderungen allein sind für einen sicheren Gebrauch noch nicht ausreichend. Ausstehende Maßnahmen werden im Kapitel~\ref{sec:Ausblick} behandelt.

Vor der Hardwareänderung liegen je zwei Motortreiberkarten auf eigenen CAN-Bussen, die separat von der SPS verwaltet werden. Dies erzeugt unnötigen Rechenaufwand für die SPS.
Um einen Betrieb sowohl über den Arduino als auch über die SPS zu ermöglichen werden, alle Komponenten auf einen CAN-Bus zusammengelegt. Dazu werden die Motortreiberkarten umadressiert und  $120 \Omega$-Abschlusswiderstände in den äußersten Steckern eingelötet.

Da die Steckerbelegung des Arduino-CAN-Shield von der Norm ISO 11 898 abweicht, wurde der entsprechende Stecker der Busleitung angepasst und ein weiterer Stecker zur Diagnose der CAN-Bus-Signale via PC angelötet.

