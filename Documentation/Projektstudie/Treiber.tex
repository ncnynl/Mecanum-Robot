% !TEX root = Projektstudie.tex
% Treiber


\section{Treiberentwicklung}
\label{sec:Treiber}

Die Schrittmotortreiber der Firma Nanotec arbeiten mit dem CANopen Protokoll. Zur Ansteuerung der Motoren sind verschieden Befehle notwendig, die im Folgenden kurz beschrieben werden. Damit der richtige Motor angesteuert wird, müssen diese Befehle an die richtige COB-ID gesendet werden. Die COB-ID setzt sich zusammen aus der 0x600 und der jeweiligen Node-ID der Schrittmotorkarte. Die Node-ID kann an der Unterseite der Schrittmotorkarten über zwei Drehschalter eingestellt werden. 

Beispiel: 
Ein Motor mit der Node-ID 3 wird über die COB-ID 0x603 angesteuert.


\subsection{Initialisierung}

Vor dem Start der Motoren muss das Leistungsteil eingeschaltet und der richtige Operationsmodus gewählt werden. 
\begin{table}[H]
    \begin{tabularx}{\textwidth}{@{}Xl@{}} \toprule
    
 
    Befehl & Code \\
    \midrule
    
    Switch on Disabled (Grundzustand)  &
    \lstinline{0x2B 0x40 0x60 0x00 0x00 0x00 0x00 0x00} \\

    Ready to Switch on  &
    \lstinline{0x2B 0x40 0x60 0x00 0x06 0x00 0x00 0x00} \\

   Switch on  &
    \lstinline{0x2B 0x40 0x60 0x00 0x07 0x00 0x00 0x00} \\
    
    Operation Enabled  &
    \lstinline{0x2B 0x40 0x60 0x00 0x0F 0x00 0x00 0x00} \\

    \bottomrule
    \end{tabularx}
    \caption{Leistungsteil einschalten}
\end{table}


\begin{table}[H]
    \begin{tabularx}{\textwidth}{@{}Xl@{}} \toprule
    
 
    Befehl & Code \\
    \midrule
    
    Velocity Mode auswählen &
    \lstinline{0x2F 0x60 0x60 0x00 0x02 0x00 0x00 0x00} \\

    \bottomrule
    \end{tabularx}
    \caption{Operationsmodus auswählen}
\end{table}


\begin{table}[h]
    \begin{tabularx}{\textwidth}{@{}Xl@{}} \toprule
    
    Befehl & Code und Beispiel \\
    \midrule
    
    Minimale Geschwindigkeit einstellen &  \ttfamily{0x23 0x46 0x60 0x01 \textcolor{blue}{0x00 0x00} 0x00 0x00} \\
    & Hier: \ttfamily{0x0000} = 0 Schritte/s \\
    
    Maximale Geschwindigkeit einstellen &  \ttfamily{0x23 0x46 0x60 0x02 \textcolor{blue}{0xA8 0x61} 0x00 0x00} \\
    & Hier: \ttfamily{0x61A8} = 25000 Schritte/s \\
    
    \textbf{Beschleunigungsrampe\footnotemark\addtocounter{footnote}{-1}:} \\
    
    
    Geschwindigkeitsänderung einstellen &  \ttfamily{0x23 0x48 0x60 0x01 \textcolor{blue}{0x20 0x4E} 0x00 0x00} \\
    & Hier: \ttfamily{0x4E20} = 20000 Schritte/s \\
    
    
    Zeitänderung einstellen &  \ttfamily{0x2B 0x48 0x60 0x02 \textcolor{blue}{0x01 0x00} 0x00 0x00} \\
    & Hier: \ttfamily{0x0001} = 1 s \\
    
    \textbf{Bremsrampe\footnotemark\addtocounter{footnote}{-1}:} \\
    
     Geschwindigkeitsänderung eintellen &  \ttfamily{0x23 0x49 0x60 0x01 \textcolor{blue}{0x20 0x4E} 0x00 0x00} \\
    & Hier: \ttfamily{0x4E20} = 20000 Schritte/s \\
    
    Zeitänderung einstellen  &  \ttfamily{0x2B 0x49 0x60 0x02 \textcolor{blue}{0x01 0x00} 0x00 0x00} \\
    & Hier: \ttfamily{0x0001} = 1 s \\

    \textbf{Bremsrampe\footnotemark:} \\
    (Für Quick-Stop) \\
    
     Geschwindigkeitsänderung eintellen &  \ttfamily{0x23 0x4A 0x60 0x01 \textcolor{blue}{0xA0 0x86} 0x00 0x00} \\
    & Hier: \ttfamily{0xA086} = 34464 Schritte/s \\
    
    Zeitänderung einstellen  &  \ttfamily{0x2B 0x4A 0x60 0x02 \textcolor{blue}{0x01 0x00} 0x00 0x00} \\
    & Hier: \ttfamily{0x0001} = 1 s \\

    \bottomrule
    \end{tabularx}
    \caption{Einstellungen im Velocity Mode}
\end{table}
\footnotetext{angegeben in Geschwindigkeitsänderung pro Zeitänderung }


\begin{table}[h]
    \begin{tabularx}{\textwidth}{@{}Xl@{}} \toprule
    
 
    Befehl & Code und Beispiel \\
    \midrule
    
    Geschwindigkeit auswählen &  \ttfamily{0x2B 0x42 0x60 0x00 \textcolor{blue}{0x10 0x27} 0x00 0x00} \\
    & Hier: \ttfamily{0x2710} = 10000 Schritte/s \\

    \bottomrule
    \end{tabularx}
    \caption{Geschwindigkeit auswählen}
\end{table}

