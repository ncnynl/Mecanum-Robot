% !TEX root = Projektstudie.tex
% API

\section{API}
Die Kommunikation mit dem Roboter findet über die UART-Schnittstelle bei einer Baudrate von 115200 Mbaud statt.
Auf dem Arduino sorgt die Library \emph{SerialCommand}\footnote{https://github.com/kroimon/Arduino-SerialCommand} für die Interpretation der Ergebnisse.

\subsection{Befehle}
Folgende Befehle sind verfügbar:
\begin{description}
\item \lstinline{@start} \\
Initialisiert und startet die Schrittmotor-Treiberkarten gemäß Kapitel~\ref{sec:Treiber}.
Der Roboter ist daraufhin fahrbereit.

\item \lstinline{@stop} \\
Führt einen Not-Stopp durch. Dabei wird der in Kapitel~\ref{sec:Treiber} beschriebene Quickstop ausgeführt.
Anschließend befinden sich die Schrittmotor-Treiberkarten im Ruhezustand und müssen erst wieder mit dem Befehl \lstinline{@start} gestartet werden.

\item \lstinline{@v v1 v2 v3 v4} \\
Über diesen Befehl werden die Soll-Radgeschwindigkeiten vorgegeben.

Anstelle von \lstinline{v1, v2, v3} und \lstinline{v4} sind durch Leerzeichen getrennte, ASCII-codierte Integer im Bereich von $-25000$ bis $25000$ (Schritte pro Sekunde) anzugeben.
\end{description}

\subsection{Fehlercodes}

\subsection{Events}
