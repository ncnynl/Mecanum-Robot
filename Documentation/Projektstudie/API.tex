% !TEX root = Projektstudie.tex
% API

\section{API}
\subsection{Befehle}
Die Kommunikation mit dem Roboter findet über serielle Schnittstelle bei 115200 baud statt.
Auf dem Arduino wird die Library \emph{SerialCommand}\footnote{https://github.com/kroimon/Arduino-SerialCommand} für die Interpretation der Ergebnisse genutzt.

Folgende Befehle sind über die serielle Schnittstelle verfügbar:
\begin{description}
\item \lstinline{@start} \\
Initialisiert und startet die Schrittmotor-Treiberkarten gemäß Kapitel~\ref{sec:Treiber}.
Der Roboter ist daraufhin fahrbereit.

\item \lstinline{@stop} \\
Führt einen Not-Stopp durch. Dabei wird der in Kapitel~\ref{sec:Treiber} beschriebene Quickstop ausgeführt.
Anschließend befinden sich die Schrittmotor-Treiberkarten im Ruhezustand und müssen erst wieder mit dem Befehl \lstinline{@start} gestartet werden.

\item \lstinline{@v v1 v2 v3 v4} \\
Über diesen Befehl werden die Soll-Radgeschwindigkeiten vorgegeben.

Für \lstinline{v1}, \lstinline{v2}, \lstinline{v3} und \lstinline{v4} sind durch Leerzeichen getrennte, ASCII-codierte Integer im Bereich von $-25000$ bis $25000$ (Schritte pro Sekunde) anzugeben.

\item \lstinline{@about} \\
Zeigt Informationen über die Version und Kompilierungszeitpunkt der Firmware sowie die Namen der Autoren.
\end{description}


\subsection{Fehlercodes}
Tritt bei der Kommunikation mit der API ein Fehler auf, führt der Roboter sofort einen Not-Stopp aus und antwortet mit einem Fehlercode sowie einer Fehlerbeschreibung.
Damit ist auch bei einer Störung der Kommunikation gewährleistet, dass der Roboter nicht außer Kontrolle gerät.
Mögliche Fehlercodes sind beispielsweise:

\lstinline{!E01: Not enough arguments supplied.}\\
\lstinline{!E02: Unknown command: [command]}\\


\subsection{Ereignisse}
Um externer Software Rückschlüsse auf den Zustand des Roboters zu ermöglichen, sendet dieser bei erfolgreichen Initialisierungen oder Befehlen Ereignisprotokolle.

Fest definierte, für die steuernde Software relevante Aktivitäten werden mit vorangestelltem Index gesendet:\\
\lstinline{@A01: All motors are ready.}\\
\lstinline{@A02: New motor speed set.}\\
\lstinline{@A03: QuickStop successful.}

Sonstige Logging-Nachrichten werden mit einer Raute versehen.
Dazu zählen aktuelle Soll-/Ist-Werte und Informationen über die Firmware, wie beispielsweise\\
\lstinline{# Debug mode activated}\\
\lstinline{# Firmware Version 1.0, compiled on Tue Jun  4 18:15:59 2013}


\subsection{C-API}
Für die Steuerung des Roboters aus der Firmware heraus werden verschiedene C-Funktionen bereitgestellt. Relevant für den Programmierer sind dabei:

\begin{description}
\item \lstinline{void robot_begin()} \\
Initialisiert die Kommunikation mit dem Roboter. Der CAN-BUS Übertragungsrate wird auf 1Mbaud festgelegt. Diese Funktion wird üblicherweise einmal in der \lstinline{setup()}-Routine des Arduinos ausgeführt.

\item \lstinline{void robot_startMotors()} \\


\item \lstinline{void robot_setMotorSpeed(int16_t speed[4])} \\
\item \lstinline{void robot_setSingleMotorSpeed(uint8_t wheel, int16_t speed)} \\
\item \lstinline{void robot_quickStop()} \\
\end{description}
