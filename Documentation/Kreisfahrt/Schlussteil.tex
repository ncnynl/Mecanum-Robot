% !TEX root = Kreisfahrt.tex
\section{Fazit}
Schnell hat sich herausgestellt, dass nur mit Kenntnis der Radgeschwindigkeiten keine genaue Positionierung möglich ist. Selbst kleinste Unebenheiten oder Verschmutzungen resultieren in starken Fahrbahnabweichungen. Für den produktiven Betrieb sind daher externe Positionsreferenzen notwendig, wie zum Beispiel eine Deckenkamera. Hinzu kommen einige mechanische Schwierigkeiten:

\begin{itemize}
    \item Die Motoren sind zu schwach dimensioniert und verlieren bei abrupten Manövern Schritte.
    \item Die Mecanum-Räder liegen zu nah am Riementrieb und verhaken sich regelmäßig mit den überstehenden Schrauben im Riemen.
    \item Das treibende Rad des Riementriebs löst sich nach längerem Gebrauch von der Welle und muss mit gezielten Hammerschlägen wieder an seine Position gebracht werden.
\end{itemize}

Dennoch wird mit dem erstellten Modell und dessen Implementierung bei einem Kreisdurchmesser von $4000 mm$ eine Reproduzierbarkeit von etwa $\pm 30 mm$ erreicht.
Durch die funktionelle Definition des Fahrwegs verlaufen die Richtungsänderungen kontinuierlich und ohne erkennbaren Ruck.
